%
% Copyright (C) 2011 Agostino De Marco
%                    <agostino dot demarco at unina dot it>
%                    Roberto Giacomelli
%                    <giaconet dot mailbox at gmail dot com>
%
%    This work may be distributed and/or modified under the
%    conditions of the LaTeX Project Public License, either
%    version 1.3 of this license or any later version.
%    The latest version of this license is in
%    http://www.latex-project.org/lppl.txt and version 1.3
%    or later is part of all distributions of LaTeX version
%    2005/12/01 or later.
%
% This work has the LPPL maintenance status `maintained'.
% 
% The Current Maintainer of this work are Agostino De Marco
% and Roberto Giacomelli
%
\documentclass{standalone}
\usepackage{lmodern}
\usepackage{pgfplots}

\tikzset{
  every pin/.style={font=\scriptsize,pin distance=4ex},
  small dot/.style={fill=gray,circle,scale=0.1}}

\begin{document}
\begin{tikzpicture}
\begin{axis}[
    % etichette
    tick label style={font=\scriptsize},
    xlabel={Rapporto in frequenza
            $\omega_\mathrm{f}/\omega_0$},
    ylabel={Amplificazione relativa
            $u_0/u_\mathrm{st}$},
    ytick={0,1,2,3,4,5},
    xtick={0,1,2,3},
    xticklabels={0,1,,},
    %
    % linee di plottaggio
    no markers,
    line width=0.3pt,
    cycle list={{black,solid}},
    %
    % dominio 2D
    samples=200,
    smooth,
    domain=0:2.5,
    xmin=0, xmax=2.5,
    ymin=0, ymax=5.0,
    %
    % dimensioni tela
    width=12cm,
    height=12cm
    ]

% horizontal help line
\draw[help lines] (axis cs:0,1) -- (axis cs:2.5,1);
% vertical help line
\draw[help lines] (axis cs:1,0) -- (axis cs:1,5);

% tracciamento curve
\addplot gnuplot {1/sqrt((1-x^2)^2+4*0.05^2*x^2)};
\addplot gnuplot {1/sqrt((1-x^2)^2+4*0.10^2*x^2)};
\addplot gnuplot {1/sqrt((1-x^2)^2+4*0.20^2*x^2)};
\addplot gnuplot {1/sqrt((1-x^2)^2+4*0.30^2*x^2)};
\addplot gnuplot {1/sqrt((1-x^2)^2+4*0.40^2*x^2)};
\addplot gnuplot {1/sqrt((1-x^2)^2+4*0.50^2*x^2)};
\addplot gnuplot {1/sqrt((1-x^2)^2+4*1.00^2*x^2)};
\addplot gnuplot {1/sqrt((1-x^2)^2+4*2.00^2*x^2)};

% tracciamento funzione dei massimi
\addplot gnuplot[dashed,domain=0:0.99]
   {1/sqrt(1-x^4)};

% etichette curve
\node[small dot,pin=30:{$\zeta_0=0{,}05$}] at
   (axis cs:1.10,4.22) {};
\node[small dot,pin=30:{$\zeta_0=0{,}10$}] at
   (axis cs:1.10,3.29) {};
\node[small dot,pin=30:{$\zeta_0=0{,}20$}] at
   (axis cs:1.10,2.05) {};
\node[small dot,pin=30:{$\zeta_0=0{,}30$}] at
   (axis cs:1.20,1.19) {};
\node[small dot,pin=30:{$\zeta_0=0{,}40$}] at
   (axis cs:1.28,0.83) {};
\node[small dot,pin=30:{$\zeta_0=0{,}50$}] at
   (axis cs:1.50,0.51) {};

\node[small dot,pin=210:{$\zeta_0=1{,}00$}] at
   (axis cs:0.52,0.79) {};
\node[small dot,pin=210:{$\zeta_0=2{,}00$}] at
   (axis cs:0.60,0.40) {};
\end{axis}
\end{tikzpicture}
\end{document}